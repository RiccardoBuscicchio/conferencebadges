%% This example is adapted from the "Business Cards for Programmers/Developers" example at 
% https://www.overleaf.com/latex/templates/business-cards-for-programmers-slash-developers/wrwgsnnmxwyg.

% Instructions are at the end of the file.

\documentclass[11pt,a4paper]{memoir}
\setstocksize{55mm}{85mm} % UK Stock size
\setpagecc{55mm}{85mm}{*}
\settypeblocksize{45mm}{75mm}{*}
\setulmargins{5mm}{*}{*}
\setlrmargins{5mm}{*}{*}
\usepackage{xcolor}

\usepackage{datatool}
%% The "database" is a comma-separated values (CSV) file.
%% The first line should contain the column headers, without space characters, e.g.
%% Name,JobTitle,Department
%%
%% If a field value contains a comma, then the field value needs to be surrounded with double quotes, e.g. 
%% John Smith,Lecturer,"School of Science, Mathematics and Engineering"
%%
%% Spreadsheet applications can usually export such a .csv file.
%%
%% If field values are expected to contain LaTeX special characters like $, &, then use \DTLloadrawdb{data}.csv instead.
\DTLloaddb{namelist}{participants.csv}

\setheadfoot{0.1pt}{0.1pt}
\setheaderspaces{1pt}{*}{*}

\checkandfixthelayout[fixed]

\pagestyle{empty}

\usepackage{graphicx}
\usepackage{pstricks}
\usepackage{auto-pst-pdf,pst-barcode}


%% These packages only for typesetting the instructions
\usepackage{listings}
\usepackage{enumitem}


\begin{document}

%% For each line in namelist (which was loaded from data.csv), 
%% output the following text (with mailmerged field values)
\DTLforeach{namelist}{%
    %% Map each column header in your .csv file to a command
	\Name=Name,%
    \Institution=Institution,
    \Email=Email,
    \Pronoun=Pronoun%
}{%%%  Start designing your output text! 
  %%%  Mailmerged field values can be inserted using the commands
  %%%  you've just mapped above.
    \begin{Spacing}{0.75}%
    \noindent
	\textbf{The most amazing conference ever}\\
    Mock University - 01-02 April 2025\\
    \rule{\textwidth}{.3mm}\\
    \begin{minipage}[t]{28mm}
        \vspace{-0mm}%
        \begin{pspicture}(21mm,21mm)
            % The MECARD format is used to exchange contact information. More information at:
            % https://www.nttdocomo.co.jp/english/service/developer/make/content/barcode/function/application/addressbook/index.html
\psbarcode{MECARD:N:\Name;EMAIL:\Email;NOTE:\Institution}{eclevel=L width=0.7 height=0.7}{qrcode}
        \end{pspicture}
    \end{minipage}
    \hspace{1mm}
    \begin{minipage}[t]{42mm}
        \vspace{-0mm}%
        \begin{flushleft}
        {\small
            \begin{Spacing}{1}%
            \textbf{\Name}
            (\footnotesize \Pronoun)\\
            \textbf{\footnotesize\Institution}
            \end{Spacing}
        }
        \end{flushleft}
    \end{minipage}
    
    \mbox{}\hfill {\scriptsize \includegraphics[height=3.0em]{logo.png}}
    \end{Spacing}
    \clearpage
}

%% Comment out this line when typesetting for final output
%%\input{instructions}


\end{document}